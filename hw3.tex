\documentclass[11pt]{article}
\usepackage{fullpage}
\usepackage{fancyhdr}
\usepackage{epsfig}
\usepackage{algorithm}
\usepackage[noend]{algorithmic}
\usepackage{amsmath,amssymb,amsthm}
\usepackage{enumerate}


% FILL IN THE SPECIFICS OF EACH HOMEWORK HERE
\newcommand{\course}{CS 364}
\newcommand{\semester}{Spring 2013}
\newcommand{\name}{Artificial Intelligence}
%%%
%%%
%%% PLEASE FILL OUT YOUR NAME AND THE HWK NUMBER
%%%
%%%
\newcommand{\hwk}{Homework \#2 Solutions}
\newcommand{\student}{Oren Shoham, Peter Fogg, Sayer Rippey}

\newtheorem{lemma}{Lemma}
\newtheorem*{lem}{Lemma}
\newtheorem{definition}{Definition}
\newtheorem{notation}{Notation}
\newtheorem*{claim}{Claim}
\newtheorem*{fclaim}{False Claim}
\newtheorem{observation}{Observation}
\newtheorem{conjecture}[lemma]{Conjecture}
\newtheorem{theorem}[lemma]{Theorem}
\newtheorem{corollary}[lemma]{Corollary}
\newtheorem{proposition}[lemma]{Proposition}
\newtheorem*{rt}{Running Time}




%%% You can ignore the following stuff, it's just for formatting purposes
\textheight=8.6in
\setlength{\textwidth}{6.44in}
\addtolength{\headheight}{\baselineskip} 
% enumerate uses a), b), c), ...
\renewcommand{\labelenumi}{\alph{enumi})}
% Sets the style for fancy pages (namely, all but first page)
\pagestyle{fancy}
\fancyhf{}
\renewcommand{\headrulewidth}{0.0pt}
\renewcommand{\footrulewidth}{0.4pt}
% Changes style of plain pages (namely, the first page)
\fancypagestyle{plain}{
  \fancyhf{}
  \renewcommand\headrulewidth{0pt}
  \renewcommand\footrulewidth{0.4pt}
  \renewcommand{\headrule}{}
}
% Changes the title box on the first page
\renewcommand\maketitle{
  \begin{center}
    \begin{tabular*}{6.44in}{l @{\extracolsep{\fill}}c r}
      \bfseries  &  & \bfseries \course ~\semester \\
      \bfseries&  & \bfseries  \hwk  \\
      \bfseries   &   &  \bfseries \student \\ 
    \end{tabular*}
\end{center} }




%%
%%
%% THE REAL STUFF STARTS HERE
%%
%%
\begin{document}
\maketitle
\thispagestyle{plain}


%%% PLEASE PLACE THE HONOR CODE AND YOUR NAME/SIGNATURE HERE
\noindent Honor Code: 

\subsection*{Part 1}
\begin{enumerate}
\item This is right.
  \begin{align*}
    P(A \text{ exec.}\ |\ B \text{ not exec.}) &= \frac{P(\text{$A$ exec.})P(\text{$B$ not exec. $\ |\ A$ exec.})}{P(\text{$B$ not exec.})} & \text{(1)} \\
    &= \frac{\frac{1}{3} \cdot 1}{\frac{2}{3}} & \text{(2)} \\
    &= \frac{1}{2}
  \end{align*}
  \begin{enumerate}[(1)]
    \item This is just an application of Bayes' theorem. Now that we know $B$ will live, we can use this to figure out $A$'s probability of being executed.
    \item We know that only one person is being executed, so if $B$ is executed then the probability of $A$ living is 1.
  \end{enumerate}
\item This explanation supposes that the probability of $A$ being executed is independent of the probabilities of the other prisoners being executed. This is wrong. What if we had learned that $B$ \emph{was} going to be executed? Then clearly $A$ would be stupid to think that they have a $\frac{1}{3}$ probability of being executed.
\end{enumerate}
\subsection*{Part 2}
\subsection*{Part 3}
\subsection*{Part 4}
\subsection*{Part 5}
\subsection*{Part 6}
\end{document}
